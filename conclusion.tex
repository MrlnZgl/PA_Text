\chapter{Conclusion}


\paragraph{Summary}
In this thesis, an optimisation approach to find material parameters for epoxies modelled with \acrshort{md} simulations, was developed. The mechanical behaviour of the material should be represented as good as possible through a constitutive model with corresponding material parameters. As constitutive model we use an elastoplastic model with a VOCE-hardening curve to describe the plastification. This model is defined with two elastic and four plastic material parameters. To monitor the quality of the material parameters, we perform simulations with the \acrshort{fe} software \name{Abaqus}. We implemented the whole optimisation algorithm in a single \name{Python} script, since \name{Abaqus} has an \acrlong{api} which enables its control via \name{Python} commands. \\
Before we start the optimisation loop, a model with specific properties is created in the preprocessing. The optimisation is started with an initial guess for the material parameters. With these parameters we perform the \acrshort{fe} simulation, and extract the resulting load reactions.
We compare the load reactions with the ones from the \acrshort{md} simulations, which we use as reference data.
The deviations are summarised in a single \acrshort{rmse} value.
This value is reduced through the numerical Nelder-Mead algorithm, which is able to optimise a scalar function in a multidimensional space. 
The algorithm adapts the material parameters and a new evaluation starts. 
To verify the algorithm, we used reference data from \acrshort{md} simulations performed by \citet{ries_deciphering_nodate} for materials with mixing ratio 6:3, where a linear tensile strain up to a maximum strain of 20\% was applied.
We performed tests with the same loading conditions for materials with mixing ratio 4:3, 6:3 and 8:3 to validate the performance of the optimisation algorithm. 
For material with mixing ratio 6:3, we applied sinusoidal strain up to a maximum amplitude of 15\%. 
We performed optimisation tests with these load parameters as tensile loading, shear loading, and finally their combination. 
In the final tests, we applied 1.5 loading cycles of sinusoidal tensile strains with 1\%, 5\% and 8\% amplitude.

\paragraph{Conclusion}
The developed optimisation approach is able to minimise the error between the reference data and the \acrlong{olr} through adaptation of the material parameters. The verification study showed, that the optimisation procedure achieves a perfect match of the load reactions for pure tensile loading through a linear strain application. However, the identified material parameters strongly deviate within the test series. We attribute this behaviour to the relatively high number of optimisation variables. Therefore, we predefined the elastic parameters $E$ and $\nu$, and optimised the plastic parameters.  In the validation study, the algorithm reliably found similar material parameters independent of the initial value combination. In addition, the \acrlong{omp} agree with the reference values found by \citet{ries_deciphering_nodate}.
The studies using sinusoidal load application provide information about the algorithm performance in different load cases. The pure tensile loading led to results of similar quality as the validation study, whereas in the shear load case issues about the optimisation of the plastic yield occurred.
Possible reasons for this behaviour were discussed in \autoref{subsec:CombiDiscussion}. In the reference data for the cyclic tests, the material properties change during the loading process, which might be explained through arising material damage. Since we disregarded damage models in our \name{Abaqus} simulation, the \acrlong{olr} show high variations from the reference data. The performed tests demonstrate, that from the tested loading procedures, optimisation of tensile loadings gives reliable results. For the other tested configurations, the results of the material parameters are still arbitrary.


\paragraph{Outlook}
The processing of shear load cases should be part of future investigations. Here, the focus should be taken on the transition from elastic to plastic material behaviour.
The tests performed in this work demonstrate open issues with the definition of the yield stress, which represents the starting point of the plastic regime. 
For an adequate representation of cyclic loading, a damage model should be included in the \name{Abaqus} model.
Because of unusual material behaviour, a subroutine would be necessary. 
Furthermore, the properties of the numerical optimisation algorithm could be part of future investigations.
Here, the initial value combinations were picked arbitrary.
Since this may lead to getting stuck in local minima, the impact of the initial value combination on the optimisation approach should be studied. In addition, the choice of the numerical algorithm could be reflected.
The sensitivity of the algorithm to the initial values could be reduced by choosing alternatives. 

% Summary: 
% - script for optimisation
% - md simulations cannot define matrial params

% - fem sim necessary to do thi (continuum based approach)
% - try to fit mechanical behaviour from md sim with const model and corresponding mat params
% - do this via optimisation
% - minimise the diff of the mechanical resp of md and fem through optimisation of mat params
% - via python script in abaqus python interface
% - use nelder mead alg and rmse to reduce diff

% Conclusion:
% 1. yes: verification showed: script minimises rmse, the load reactions match perfectly
% but2: no unique solution for the mat params --> fix elastic params, since only one point in the elastic regime

% 2. yes in validation study: if elast params are fixed, the other params can be determined independent of the initial val combi, except gamma but gamma has only a small effect on the trend of the hardening curve --> not important 

% 3.: verification and validaiton: for linear tensile load works fine 
% tensile: tensile as sin function works too
% shear: yield stress runs into limit in all tests --> difiiculties in finding limit between el and plastic domain
% tensile and shear combined: problems of shear are transferred, and elastic params are not similar --> not possible to match obth curves at a time --> error much higher than before, plot stress strain tens 
% cyclic: works just until unloading stars, because of material damage, cannot represent mat damage, because elastic behaviour is influenced too --> have to write own usersubroutine 

% conclusion: 
% userfriednly: only input file must be adapted 
% everything in one script, automatic data storage
% dynamic adaption for various loading combinations (mult load params, load cases, initial value combinations)
% optimisation script generally works for tensile load application for different load parameters
% for shear load problems at differnetiation between el and plast regime 
% for cyclic damage inclusion
% md data maybe not perfectly suitable

% outlook: 
% - for shear load improvement of differnetiation between el and plast regime 
% - for cyclic damage inclusion
% - more equivalent distribution of reference data points in the domain --> more elastic points
% - ensrue reliability of relaxation procedure for shear loading



