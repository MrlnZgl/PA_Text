\chapter{Introduction}

Epoxies are wonderful because of XXX. Their applications are XXXX. Joining technology: adhesive joining because of XX. properties of adhesive joints are XXX. However, their mechanical properties are not completely studied so far (QUELLE). To study their properties, multiple technologies are used (QUELLE), VLT AUFLISTUNG WELCHE. 

Adhesive bonding is an increasingly popular joining technique due to its applicability for composite materials, and their beneficial loading properties \cite{campilho_extended_2011}, \cite{pramanik_joining_2017}. The extension of usage in further applications depends on a profound understanding of their material behaviour. For investigations on atomistic level \acrfull{md} is a widely used approach  \cite{ries_mechanical_2024}. It is capable to model the curing process, XX and XX and the mechanical behaviour. However, investigations on atomistic level require high resolutions in space and time which cause high computational costs. Due to this issue, only small-scaled simulations are possible. If the findings are extrapolated on larger dimensions, a sufficient transferability must be ensured (SCHLECHTE FORMULIERUNG). Especially for the mechanical behaviour this might be challenging. To transfer the governed results to real-life engineering applications, adequate engineering quantities must be extracted. For the description of the mechanical behaviour, the translation takes place through material parameters. They have specific values for every material. If the material parameters are known, the material behaviour can be calculated through functional relations so-called Constitutive models. Hence, an adaption to the current problem is easily possible. However, with \acrshort{md} simulations a direct extraction of the material parameters is not possible. This is due to the fact, that the \acrshort{md} simulation is a method on an atomistic level, whereas material parameters are defined for a continuum-based perspective (QUELLE). Continuum-based means XXX Consequently, a continuum-based method must be used, to identify the material parameters. The \acrfull{fem} is such a method. IWELCHE VORTEILE VON FEM GRUNDLAGEN DEFORMATIONENN GEHEN SCHNELLE. With \acrshort{fem} analysis, the loading process on arbitrary bodies can be simulated quite fast (QUELLE). In the \acrshort{fem} analysis, the mechanical responses for an applied load case is determined through the previously defined constitutive model and the material parameters. Conversely, for known mechanical behaviour in an applied load case, and a defined constitutive model, the quality of the material parameters can be evaluated retrospectively. 
These properties make \acrshort{fem} a powerful tool for a continuum-based analysis of the mechanical behaviour of epoxys. Nevertheless, the material properties found by \acrshort{md} simulations need to be considered too. Therefore, a procedure is required which is able to define material parameters for epoxys, based on the findings from \acrshort{md} analysis. In addtion the procedure should be easy to handle and fast.
\paragraph{Scope of this work}
In this work, such a procedure is developed, to determine material parameters via \acrshort{fem} analysis considering the material properties detected via \acrshort{md} simulations. Therefore, the mechanical responses in the simulations methods need to be compared. Based on their conformity, the material parameters of the \acrshort{fem} simulation are evaluated. This leads to an optimisation problem, where optimum material parameters should be found to imitate the mechanical behaviour detected in the \acrshort{md} simulations as good as possible. The developed process will be used to answer the following research questions: 1. Can we qualitatively replicate the mechanical responses generated by \acrshort{md} simulations, using \acrshort{fem} analysis? 2. Can we find material parameters which lead to quantitatively matching mechanical responses? 3. How stable and robust is the process? To this end, we first explain the general characteristics of \acrshort{md} and \acrshort{fem} (\autoref{sec: MDBasics} and \autoref{sec: FEMBasics}), and present the selected tools to implement the optimisation procedure (\autoref{sec: AbaqusBasics} and \autoref{subsec: numericaloptimisation}). Then, the setup of the developed optimisation process is outlined (\autoref{chap: modelsAndMethods}). Afterwards, we validate the script and discuss possible issues. The capability of the code is tested through multidimensional load applications and cyclic loadings. Finally, we discuss the dependency of the optimisation performance of the loading conditions and detect possible improvements.



% MUSS IN EINLEITUNG GANZ ZUM SCHLUSS
% The developed process should avoid data transfer between different programs for a higher user-friendliness. Furthermore, it should produce reliable results within a limited timeframe. 

% - epoxys
% - MD simulation- not possible to determine Material parameters
% - MD very costly
% - if only mat params interesting is MD too high effort
% - other procedure: fast and reliable results and easy to use
% - general description of Mat behaviour through Mat Params
% - useful for more general investigations of epoxy behaviour --> more specific applications
% - userfriendly in one application



% questions of this work: 
% - can we implement an optimisation procedure which uses results of MD simulaitons to find material parameters?
% - gives the process results for the Material params which lead to mathcing load reactions?
% - Is the process independent of the initial values?
% - Is the process independent of the loading directions and the trend of the applied load?
% (- is it possible find unique mat params which represent the mat behaviour in different loading processes?)
