\chapter{Basics}
This chapter lays the foundations to understand this thesis. First \acrfull{md} is simulation method is introduced. Afterwards the constitutive model is presented. In the last section the used scripting tool with its plug-in for periodic boundary conditions is explained.  

\section{Molecular dynamics approach}
Adhesive joints are important because of XXX. The extension of usage in further applications depends on a profound understanding of their material behaviour. For investigations on atomistic level \acrfull{md} is a widely used approach QUELLE MAx. 

- solve newtons equation for each atom from interatiocns with neighbours
- modeled via potentials
- non-bonded interactions too
- timestep integration in femtosecond
- only small timescales possible
- hihg numerical effort --> small dimensions
- high surface to volume ratio --> bad 
- periodic bc eliminate free surface effect
- representative volume (Hill) which is strucutrally typical for the whole mixture
- periodic bc sorgen dafür dass randeffekte keine einfluss haben
- beschränken volumen so, dass es sich verhält als wäre es in einem unendlichen volumen enthalten
- parallel flächen bleiben parallel


The interphase between the adhesive and adherend phase is particularly important \cite{roche_measurement_1994}, since the mechanical properties of this region depend on locally varying mixing ratios between resin and hardener (resin:hardener) \cite{ries_deciphering_nodate}, \cite{garifullin_dependence_2019}. The investigation of the locally variating mixing ratio is possible through \acrshort{md} simulations \cite{dotschel_reactive_2026}. \acrshort{md} simulations allow building samples with prescribed properties followed by deformation tests to study the material behaviour \cite{buyukozturk_structural_2011}. 
This work focusses on the investigations by \citet{ries_deciphering_nodate} who studied samples with different mixing ratios in the described way. Their performed deformation tests build the motivation for the here developed optimization process. \citet{ries_deciphering_nodate} ran uniaxial tensile tests loading a sample with a linear strain up to a maximum value of 20 \%. The test sample is constrained by periodic boundary conditions (section XX) which allow lateral contractions. To record the stress-strain response without viscous amounts they developed a procedure to approximate the quasi-static material response. They performed this test with mixing ratios of 4:3, 6:3 and 8:3. The corresponding stress-strain responses are used in the verification and validation of the code developed in this work. 


hier noch pbc einbauen

\section{Finite Element Method}

\section{Periodic boundary conditions}

\section{VOCE-Model}

- used in MD-Simulation too
- necessary for comparable results


\section{Optimization algorithm} \label{optimizationAlgorithm}

To find the values of material parameters fitting best the material behaviour measured in the MD-simulation a mathematical formulation is necessary. This leads to an optimization problem, where a calculated error (see \autoref{sec: errorCalculation}), defined as an objective function of the material parameter values, should be minimized. To solve this optimization problem various mathematical algorithms are available. We decided to use the Nelder-Mead algorithm, which is a widely used gradient-free optimization algorithm \cite{gao_implementing_2012}. In a gradient-free algorithm the derivates of the function are not included in the process. Our objective function is based on results from a finite-element-analysis, which makes it impossible to determine its derivatives directly. Therefore, only gradient-free algorithms come into account. In addition, ignoring the derivatives saves significant computational costs, which leads to fast convergence times \cite{pham_comparative_2011}. Due to its simple structure the algorithm is a standard feature in many numerical libraries \cite{singer_efficient_2004}. In \name{python} it is available in the SciPy.optimize minimize -function. In \autoref{sce: optimizationCode} the function call is described in detail. Here we focus on the procedure of the algorithm. The algorithm is capable to find a local minimum of a scalar function depending on $n$ optimization variables. In this work the optimization variables are the material parameters. The definition of the objective function can be found in \autoref{sec: methodTheory}. Assuming the objective function is known, the first step is to create $n+1$ points $\mathbf{P}$ in an $n$-dimensional space. In the initial step of the algorithm the position of the points has to be determined. This is done by an initial guess $\hat{x}$ for every optimization variable value. To process six optimization variables the initial guess would look like
\begin{gather*}
    \mathbf{\hat{x}} = [\hat{x}^0, \hat{x}^1, \hat{x}^2, \hat{x}^3, \hat{x}^4, \hat{x}^5] \\
    \text{with } \hat{x}^i \equiv \text{initial guess of the $i$-th optimization variable}
\end{gather*}

Based on this the initial points $\mathbf{\hat{P}_i}$ are constructed. The first one is defined as $\mathbf{\hat{P}_1} = \mathbf{\hat{x}}$. For the other points the value of one variable in the initial guess is changed each. The points result in an $n$ dimensional simplex. In the next step the function values corresponding to the points $\mathbf{P_i}$ are evaluated and sorted by size. The highest function value $y_h$ thus maps the worst value combination $\mathbf{P_h}$ of the optimization parameters. Afterwards a centroid of all points of the simplex except $\mathbf{P_h}$ is determined. Now there are four possible operations to improve the position of $\mathbf{P_h}$. Reflection and expansion of $\mathbf{P_h}$ at the centroid are the first two. Before the new point $\mathbf{P^{*}}$ is positioned the corresponding function value needs to be evaluated. Only if $y^{*}$ is smaller than $y_l$, $\mathbf{P^{*}}$ is set as new point $\mathbf{P_i}$ in the simplex. If $y^{*}$ is larger than $y_l$, the new point is even worse than $\mathbf{P_h}$. Therefore, the operations contraction or shrinking have to be performed. They should find a position $\mathbf{P^{**}}$ between $\mathbf{P_h}$ and its reflection $\mathbf{P^{*}}$ which leads to a better function value $y^{**}$. This needs multiple iterations because for every guess $\mathbf{P^{**}}$ the function has to be evaluated. Only when a better position $\mathbf{P_h}$ is replaced by $\mathbf{P^{*}}$ or $\mathbf{P^{**}}$ and the algorithm starts again with the new simplex \cite{nelder_simplex_1965}. Therefore multiple function evaluations are necessary during one iteration of the optimization. If the variations of the functions values $y_i$ fall under a certain limit, the minimum with its corresponding parameter values is found. To ensure a successful search the initial simplex should be scaled regularly \cite{baudin_nelder-mead_nodate} which is possible through a regular distribution of the points $\mathbf{\hat{P}_i}$ in space. This can be difficult if the values of the optimization variables differ much in size. Therefore, it is necessary to normalize the variable values within the range of 0 to 1. 



% - have n varibales --> here 6 
% - create n+1 points in a n-dimensional space
% - function y 
% - P_i are the n1 points
% - y(P_i) are the function values from that we create a simplex
% - determine y_low and y_high (lowest and highset function values)
% - build centroid y_c between all points except y_high
% - then three options to replace y_high (which is the worst value)
% reflection: refelct y_high at y_c (Formel aus paper) is called y*
% - i y* is better than y_high we replace it 
% - if y* is better than y_low (we found a new minimum) we expand y* in the same direciton to y^{**}
% if y++ is better than y_low we replace y_h by y++
% - if not we reaplce y_h by y*
% contraction: if y* is is higher than all other y's (so we just found a value whcih is still the maximum), we chosse p_h or P* to be the new P_h 
% - then use contraction coefficient to find P** (inside or outside contraction)
% - if y** worse than y_high or y* --> shrink towards P_l 
% - stopping cruíterion: error smaller than defined value
% - simplex should not become too small compared to the curvature of the surface -> leads to small curvatures which lead to high variance in the estimates without finding accurate minimum
% - create intial simplex with x_0 as input --> macht SciPy irgendwie, angelbich nach irgendeiner logik 
% - intiial guess ist P_1(rray mit 6 variablen) dann wird durch den array iteriert und jeweils ein wert verändert und daraus der näcshte Oonkt P_i erzeugt -> jeweils nur veränderung einer variablen. 
% - scipy has dynmaic scaling how to variate the intial values



\section{ABAQUS PDE}
\subsection{EasyPBC Plug-In}

