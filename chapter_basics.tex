\chapter{Basics}
This chapter lays the foundations to understand this thesis. First \acrfull{md} is simulation method is introduced. Afterwards the constitutive model is presented. In the last section the used scripting tool with its plug-in for periodic boundary conditions is explained.  

\section{Molecular dynamics}
Adhesive joints are important because of XXX FÜR POLYMERE. The extension of usage in further applications depends on a profound understanding of their material behaviour. For investigations on atomistic level \acrfull{md} is a widely used approach  \cite{ries_mechanical_2024}. From the interactions with neighbouring atoms Newtons equation is solved for every atom. These interactions are modeled via potentials. Non-bonded interactions like van der Waals potentials are considered within a cutoff radius. The total potential energy of the system helps to identify the acting forces and accelerations of each particle. To follow the movements of the particles time integration is necessary. Usual are small time step sizes in femtoseconds what makes only small time scales possible with suitable computational costs \cite{ries_mechanical_2024}. Similar restriction holds for the system size, due to the increasing number of interactions with increasing element dimensions. However, small dimensions lead to large surface-to-volume ratios which result in significant free surface effects. To avoid them \acrfull{pbc} are used. They constrain the simulated element as if it is integrated in an infinitely large volume. To achieve this the boundaries of the system close themselves, which results in a simulation of an infinitively long concatenation of the same element in each direction \cite{gorbunov_periodic_2022}. Regarding the particle tracking, a particle which leaves the system at one surface enters the system then at the opposite surface. For the deformation of the whole system is restricted in a way that parallel surfaces remain parallel during loading procedures. With these adaptions the results from \acrshort{md} simulations can be transferred to a larger system. Thus \acrshort{md} simulations allow building samples with prescribed properties followed by deformation tests to study the material behaviour \cite{buyukozturk_structural_2011}. To deduce general stress-strain curves from \acrshort{md} simulations, a mathematical expression is required. This is done by constitutive models which describe the relation between stresses and strains \cite{mergheim_lecture_nodate}. Polymers are usually modeled with elastoplastic models. They show elastic behaviour until a yield strength is reached. Then the plastification begins which can be described by various hardening models depending on the material characteristics \cite{mergheim_lecture_nodate}.


% . Constitutive model: 
% - equations which describe stress-strain behaviour of material
% - multiple models depending on the material behaviour
% - try to fit the material curve 
% - here voce model 
% - hyperelastic and plastic

% VOCE model hier beschreiben
% - constitutive material model
% - hier verwendet um material parameter zu bestimmen
% - einzelne parameter beschreiben
% - bild mit einfluss verschidener parameter auf hardening curve 
% - hyperelastic material weil high strain rate 
% - elastoplastisch? 
% - auswertungsmethode beschreiben von MAx: durch quasie static kann viscosity ingonriert werden --> wird gewartet bis viskose reaktion aufgehört hat
%  KEINE SINNVOLLEN QUELLEN DAZU GEFUNDEN --> MAX FRAGEN  

% - solve newtons equation for each atom from interatiocns with neighbours
% - modeled via potentials
% - non-bonded interactions too
% - timestep integration in femtosecond
% - only small timescales possible
% - hihg numerical effort --> small dimensions
% - high surface to volume ratio --> bad 
% - periodic bc eliminate free surface effect
% - representative volume (Hill) which is strucutrally typical for the whole mixture
% - periodic bc sorgen dafür dass randeffekte keine einfluss haben
% - beschränken volumen so, dass es sich verhält als wäre es in einem unendlichen volumen enthalten
% - parallel flächen bleiben parallel --> knoten an gegenüberliegenden Flächen sind miteinander verbunden (bei MD vmtl anders implementiert)
% - particle which leaves RVE over one surface kommt es auf der ggü liegenden Fläche wieder rein an der stelle an der es theoritsch im nächsten Volume landen würde
% - wie wird deformation bechsränkt? --> parallel flcäehn 
% - ergebnisse sollen vewendet werden, dher muss modelaufbau übernommen werden: Würfel, PBC, Loading




% The interphase between the adhesive and adherend phase is particularly important \cite{roche_measurement_1994}, since the mechanical properties of this region depend on locally varying mixing ratios between resin and hardener (resin:hardener) \cite{ries_deciphering_nodate}, \cite{garifullin_dependence_2019}. The investigation of the locally variating mixing ratio is possible through \acrshort{md} simulations \cite{dotschel_reactive_2026}. 


This work focusses on the investigations by \citet{ries_deciphering_nodate} who studied the curing and deformation properties of epoxy through \acrshort{md} simulation. They developed models with numerous mixing ratios of resin and hardener. Their performed deformation tests build the motivation for the here developed optimization process. \citet{ries_deciphering_nodate} ran uniaxial tensile tests loading a sample with a linear strain up to a maximum value of 20 \%. The test sample is constrained by \acrshort{pbc} which allow lateral contractions. To record the stress-strain response without viscous amounts they developed a procedure to approximate the quasi-static material response. Then only elastic and plastic reactions are considered. To describe the elastic material behaviour they used the Neo-Hookean hyperelasticity model. The plastic reactions are modeled via the VOCE-model which defines the stresses during the hardening process through (Zitat voce)

\begin{equation}
    \sigma = \sigma_0 + \alpha(1 - \text{exp}(-\beta \varepsilon_{pl})) + \gamma \varepsilon_{pl}
\end{equation}
\begin{gather*}
    \sigma_0: \text{ Yield stress} \\
    \varepsilon_{pl}: \text{ Plastic strain} \\
    \alpha, \beta,  \gamma: \text{ Hardening parameters}
\end{gather*}
Together with the elastic material parameters Young's modulus $E$ and Poison's ratio $\nu$, six constitutive parameters are available to fit the stress-strain pairs measured through \acrshort{md} simulation. Their values are calculated with a minimization algorithm. The detailed procedure is described in \cite{ries_deciphering_nodate}. The procedure of \citet{ries_deciphering_nodate} is important since their data are used for the model assessment of the optimization procedure developed in this work. A detailed description of the optimization setup is given in \autoref{chap: modelsAndMethods}. In the verification studies the optimization procedure is tested with mixing ratios 4:3, 6:3 and 8:3. To evaluate its performance the optimized material parameters values are compared to the material parameter values governed by \citet{ries_deciphering_nodate}. Though a valid comparison is only possible if first the stress strain data are collected under similar loading conditions. And second the same constitutive model is used to govern the material parameter values. Thus, a detailed understanding of the methods used by \citet{ries_deciphering_nodate} is necessary, since they are adopted to the simulation process used in this work.  




% with parameters
% \begin{gather}
%     C_{10} = \frac{E}{4(1+ \nu)} \\
%     D_1 = \frac{3(1-2\nu)}{2E}
% \end{gather}
% \begin{equation*}
%     E: \text{ Youngs modulus} \hspace{1.5cm} \nu: \text{ Poisson ratio}
% \end{equation*}




- jz mit FEM weil XXXX


\section{Finite Element Method}
\section{ABAQUS PDE}
\subsection{EasyPBC Plug-In}



\section{Optimization algorithm} \label{optimizationAlgorithm}

To find the values of material parameters fitting best the material behaviour measured in the MD-simulation a mathematical formulation is necessary. This leads to an optimization problem, where a calculated error (see \autoref{sec: errorCalculation}), defined as an objective function of the material parameter values, should be minimized. To solve this optimization problem various mathematical algorithms are available. We decided to use the Nelder-Mead algorithm, which is a widely used gradient-free optimization algorithm \cite{gao_implementing_2012}. In a gradient-free algorithm the derivates of the function are not included in the process. Our objective function is based on results from a finite-element-analysis, which makes it impossible to determine its derivatives directly. Therefore, only gradient-free algorithms come into account. In addition, ignoring the derivatives saves significant computational costs, which leads to fast convergence times \cite{pham_comparative_2011}. Due to its simple structure the algorithm is a standard feature in many numerical libraries \cite{singer_efficient_2004}. In \name{python} it is available in the SciPy.optimize minimize -function. In \autoref{sce: optimizationCode} the function call is described in detail. Here we focus on the procedure of the algorithm. The algorithm is capable to find a local minimum of a scalar function depending on $n$ optimization variables. In this work the optimization variables are the material parameters. The definition of the objective function can be found in \autoref{sec: methodTheory}. Assuming the objective function is known, the first step is to create $n+1$ points $\mathbf{P}$ in an $n$-dimensional space. In the initial step of the algorithm the position of the points has to be determined. This is done by an initial guess $\hat{x}$ for every optimization variable value. To process six optimization variables the initial guess would look like
\begin{gather*}
    \mathbf{\hat{x}} = [\hat{x}^0, \hat{x}^1, \hat{x}^2, \hat{x}^3, \hat{x}^4, \hat{x}^5] \\
    \text{with } \hat{x}^i \equiv \text{initial guess of the $i$-th optimization variable}
\end{gather*}

Based on this the initial points $\mathbf{\hat{P}_i}$ are constructed. The first one is defined as $\mathbf{\hat{P}_1} = \mathbf{\hat{x}}$. For the other points the value of one variable in the initial guess is changed each. The points result in an $n$ dimensional simplex. In the next step the function values corresponding to the points $\mathbf{P_i}$ are evaluated and sorted by size. The highest function value $y_h$ thus maps the worst value combination $\mathbf{P_h}$ of the optimization parameters. Afterwards a centroid of all points of the simplex except $\mathbf{P_h}$ is determined. Now there are four possible operations to improve the position of $\mathbf{P_h}$. Reflection and expansion of $\mathbf{P_h}$ at the centroid are the first two. Before the new point $\mathbf{P^{*}}$ is positioned the corresponding function value needs to be evaluated. Only if $y^{*}$ is smaller than $y_l$, $\mathbf{P^{*}}$ is set as new point $\mathbf{P_i}$ in the simplex. If $y^{*}$ is larger than $y_l$, the new point is even worse than $\mathbf{P_h}$. Therefore, the operations contraction or shrinking have to be performed. They should find a position $\mathbf{P^{**}}$ between $\mathbf{P_h}$ and its reflection $\mathbf{P^{*}}$ which leads to a better function value $y^{**}$. This needs multiple iterations because for every guess $\mathbf{P^{**}}$ the function has to be evaluated. Only when a better position $\mathbf{P_h}$ is replaced by $\mathbf{P^{*}}$ or $\mathbf{P^{**}}$ and the algorithm starts again with the new simplex \cite{nelder_simplex_1965}. Therefore multiple function evaluations are necessary during one iteration of the optimization. If the variations of the functions values $y_i$ fall under a certain limit, the minimum with its corresponding parameter values is found. To ensure a successful search the initial simplex should be scaled regularly \cite{baudin_nelder-mead_nodate} which is possible through a regular distribution of the points $\mathbf{\hat{P}_i}$ in space. This can be difficult if the values of the optimization variables differ much in size. Therefore, it is necessary to normalize the variable values within the range of 0 to 1. 



% - have n varibales --> here 6 
% - create n+1 points in a n-dimensional space
% - function y 
% - P_i are the n1 points
% - y(P_i) are the function values from that we create a simplex
% - determine y_low and y_high (lowest and highset function values)
% - build centroid y_c between all points except y_high
% - then three options to replace y_high (which is the worst value)
% reflection: refelct y_high at y_c (Formel aus paper) is called y*
% - i y* is better than y_high we replace it 
% - if y* is better than y_low (we found a new minimum) we expand y* in the same direciton to y^{**}
% if y++ is better than y_low we replace y_h by y++
% - if not we reaplce y_h by y*
% contraction: if y* is is higher than all other y's (so we just found a value whcih is still the maximum), we chosse p_h or P* to be the new P_h 
% - then use contraction coefficient to find P** (inside or outside contraction)
% - if y** worse than y_high or y* --> shrink towards P_l 
% - stopping cruíterion: error smaller than defined value
% - simplex should not become too small compared to the curvature of the surface -> leads to small curvatures which lead to high variance in the estimates without finding accurate minimum
% - create intial simplex with x_0 as input --> macht SciPy irgendwie, angelbich nach irgendeiner logik 
% - intiial guess ist P_1(rray mit 6 variablen) dann wird durch den array iteriert und jeweils ein wert verändert und daraus der näcshte Oonkt P_i erzeugt -> jeweils nur veränderung einer variablen. 
% - scipy has dynmaic scaling how to variate the intial values





